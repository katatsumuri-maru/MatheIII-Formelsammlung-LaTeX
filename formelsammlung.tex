\documentclass[11pt,a4paper,landscape]{article}
\pagestyle{empty}
\usepackage[landscape, left=0.5cm, right=0.5cm, top=0.5cm, bottom=0.5cm]{geometry}
\usepackage{paracol}
\usepackage{amsmath,amssymb}

\begin{document}
\begin{paracol}{3}

\section{Funktionen in $\mathbb{R}^n$}
\subsection{Metrik}

Eine Funktion \( d: \mathbb{R}^n \times \mathbb{R}^n \to \mathbb{R}_0^+ \) mit \( \forall x, y, z \in \mathbb{R}^n \):

\noindent
\textbf{Positive Definitheit}

$
    d(x, y) \geq 0 \quad \text{und} \quad d(x, y) = 0 \iff x = y
$

\noindent
\textbf{Symmetrie}
$
    d(x, y) = d(y, x)
$

\noindent
\textbf{Dreiecksungleichung}
$
    d(x, y) \leq d(x, z) + d(z, y)
$

\noindent
\textbf{Triviale Metrik:}

$
d(x,y) = \begin{cases}
1 &\text{falls } x \neq y \\
0 &\text{falls } x = y \\
\end{cases}
$

\noindent
\textbf{Französische Eisenbahnmetrik:}

$
d(x,y) = \begin{cases}
|x-y| &\text{falls } \exists \lambda \in \mathbb R : x = \lambda y \\
|x| + |y| &\text{sonst} \\
\end{cases}
$

\subsection{Normen in $\mathbb R ^n$}

Eine Funktion $||.|| : \mathbb R^n \to \mathbb R^+_0$ mit

\noindent
\textbf{Positive Definitheit}
$
 ||x|| \ge 0 \land ||x|| = 0 \iff x = 0
$

\noindent
\textbf{Homogenität}
$
 ||\lambda x|| = |\lambda| ||x||
$

\noindent
\textbf{Dreiecksungleichung}
$
 ||x+y|| \le ||x|| + ||y||
$

\textbf{$l_1$-Norm}
$
||x||_1 = |x_1| + \dots + |x_n|
$

\textbf{$l_p$-Norm}
$
||x||_p = \sqrt[p]{x_1^p+\dots+x_n^p}
$

\textbf{Maximum-Norm}
$
||x||_\infty = \max\{ |x_i| : i=1, \dots, n\}
$

\noindent
\textbf{Satz}
Jede Norm induziert über $d(x,y) = ||x-y||$ eine Metrik.

\subsection{Bilinearform}
Eine Funktion $\langle.,. \rangle_A : \mathbb R^n \times R^n \to \mathbb R$ heißt billinearform, wenn sie in beiden Argumenten linear ist:

$
\langle \alpha x + \beta y, z\rangle_A = \alpha \langle x,z \rangle_A + \beta \langle y,z \rangle_A
$

$
\langle z, \alpha x + \beta y \rangle_A = \alpha \langle z,x \rangle_A + \beta \langle z,y\rangle_A
$

\noindent
Weitere, mögliche Eigenschaften:

\textbf{Symmetrie}: $\forall x,y \in \mathbb R^n :\langle x,y \rangle_A = \langle y, x \rangle_A$

\textbf{Positive Definitheit}: $\forall x \in \mathbb R^n \land x \neq 0: \langle x,x \rangle_A > 0$

\noindent
Jede Quadratische Matrix $B$ erzeugt eine Bilinearform über $x^T B y$

\switchcolumn
\subsection{Topologie in $\mathbb R^n$}

\textbf{Offene Kugel mit Radius r um a:} 

$U_R(a) := \{x \in \mathbb R^n: ||x-a|| < r \}$

\noindent
\textbf{Innerer Punkt $a \in A \subseteq \mathbb R^n$:} 
$\exists \varepsilon > 0: U_\varepsilon(a) \subseteq A$

\noindent 
\textbf{Offenes Intervall:} Jeder Punkt ist innerer Punkt: 

$\forall a \in A : \exists \varepsilon > 0: U_\varepsilon(a) \subseteq A$
\indent bsp: $\mathbb R^n, \{\}, U_r(a)$

\noindent
\textbf{Abgeschlossenes Intervall:}
Komplement ist offen

bsp: $\mathbb R^n, \{ \}$

\noindent
\textbf{Beschränkt:}
$\exists M \in \mathbb R: A \subset U_M(0)$

\noindent
\textbf{Kompakt:} Abgeschlossen und beschränkt

\subsection{Folgen in $\mathbb{R}^n$}

\textbf{Konvergenz:}

$\exists g \in \mathbb R^n :\forall \varepsilon > 0: \exists M \in \mathbb N: \forall m \ge M: ||a_m -g|| < \varepsilon$

\noindent
\textbf{Mehrdimensionale Konvergenz:} Konvergiert genau dann, wenn alle Komponenten gegen den jeweiligen Wert konvergieren.

\noindent
\textbf{Cauchyfolge:}

$\forall \varepsilon > 0: \exists M \in \mathbb N; \forall s,m \ge M: ||a_s - a_m|| < \varepsilon$

\noindent
\textbf{Banachraum:}
Jede Cauchyfolge Konvergiert und die Metrik wird durch eine Norm induziert

Bsp: $R^n$ mit $l_p$-Norm

\subsection{Funktionen}
\textbf{Graph von $f: \mathbb R^n \to \mathbb R^m$:}

$\Gamma_f := \{(x_1, \dots, x_n , z_1, \dots, z_m \in) D \times \mathbb R ^m: z_i = f_i(x_1, \dots, x_n)\}$

\noindent
\textbf{Höhen/Niveaulinie von $f: D \subseteq \mathbb R^2 \to \mathbb R$}

$H_c(f) := \{(x,y) \in D \mid f(x,y) = c\}$

\noindent \textbf{Niveaufläche von $f:D \subseteq \mathbb R^n \to \mathbb R$:}

$H_c(f) := \{(x_1,\dots, x_n) \in D \mid f(x_1, \dots, x_n) = c\}$
\switchcolumn
\end{paracol}
\end{document}